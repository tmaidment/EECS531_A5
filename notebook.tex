
% Default to the notebook output style

    


% Inherit from the specified cell style.




    
\documentclass[11pt]{article}

    
    
    \usepackage[T1]{fontenc}
    % Nicer default font (+ math font) than Computer Modern for most use cases
    \usepackage{mathpazo}

    % Basic figure setup, for now with no caption control since it's done
    % automatically by Pandoc (which extracts ![](path) syntax from Markdown).
    \usepackage{graphicx}
    % We will generate all images so they have a width \maxwidth. This means
    % that they will get their normal width if they fit onto the page, but
    % are scaled down if they would overflow the margins.
    \makeatletter
    \def\maxwidth{\ifdim\Gin@nat@width>\linewidth\linewidth
    \else\Gin@nat@width\fi}
    \makeatother
    \let\Oldincludegraphics\includegraphics
    % Set max figure width to be 80% of text width, for now hardcoded.
    \renewcommand{\includegraphics}[1]{\Oldincludegraphics[width=.8\maxwidth]{#1}}
    % Ensure that by default, figures have no caption (until we provide a
    % proper Figure object with a Caption API and a way to capture that
    % in the conversion process - todo).
    \usepackage{caption}
    \DeclareCaptionLabelFormat{nolabel}{}
    \captionsetup{labelformat=nolabel}

    \usepackage{adjustbox} % Used to constrain images to a maximum size 
    \usepackage{xcolor} % Allow colors to be defined
    \usepackage{enumerate} % Needed for markdown enumerations to work
    \usepackage{geometry} % Used to adjust the document margins
    \usepackage{amsmath} % Equations
    \usepackage{amssymb} % Equations
    \usepackage{textcomp} % defines textquotesingle
    % Hack from http://tex.stackexchange.com/a/47451/13684:
    \AtBeginDocument{%
        \def\PYZsq{\textquotesingle}% Upright quotes in Pygmentized code
    }
    \usepackage{upquote} % Upright quotes for verbatim code
    \usepackage{eurosym} % defines \euro
    \usepackage[mathletters]{ucs} % Extended unicode (utf-8) support
    \usepackage[utf8x]{inputenc} % Allow utf-8 characters in the tex document
    \usepackage{fancyvrb} % verbatim replacement that allows latex
    \usepackage{grffile} % extends the file name processing of package graphics 
                         % to support a larger range 
    % The hyperref package gives us a pdf with properly built
    % internal navigation ('pdf bookmarks' for the table of contents,
    % internal cross-reference links, web links for URLs, etc.)
    \usepackage{hyperref}
    \usepackage{longtable} % longtable support required by pandoc >1.10
    \usepackage{booktabs}  % table support for pandoc > 1.12.2
    \usepackage[inline]{enumitem} % IRkernel/repr support (it uses the enumerate* environment)
    \usepackage[normalem]{ulem} % ulem is needed to support strikethroughs (\sout)
                                % normalem makes italics be italics, not underlines
    

    
    
    % Colors for the hyperref package
    \definecolor{urlcolor}{rgb}{0,.145,.698}
    \definecolor{linkcolor}{rgb}{.71,0.21,0.01}
    \definecolor{citecolor}{rgb}{.12,.54,.11}

    % ANSI colors
    \definecolor{ansi-black}{HTML}{3E424D}
    \definecolor{ansi-black-intense}{HTML}{282C36}
    \definecolor{ansi-red}{HTML}{E75C58}
    \definecolor{ansi-red-intense}{HTML}{B22B31}
    \definecolor{ansi-green}{HTML}{00A250}
    \definecolor{ansi-green-intense}{HTML}{007427}
    \definecolor{ansi-yellow}{HTML}{DDB62B}
    \definecolor{ansi-yellow-intense}{HTML}{B27D12}
    \definecolor{ansi-blue}{HTML}{208FFB}
    \definecolor{ansi-blue-intense}{HTML}{0065CA}
    \definecolor{ansi-magenta}{HTML}{D160C4}
    \definecolor{ansi-magenta-intense}{HTML}{A03196}
    \definecolor{ansi-cyan}{HTML}{60C6C8}
    \definecolor{ansi-cyan-intense}{HTML}{258F8F}
    \definecolor{ansi-white}{HTML}{C5C1B4}
    \definecolor{ansi-white-intense}{HTML}{A1A6B2}

    % commands and environments needed by pandoc snippets
    % extracted from the output of `pandoc -s`
    \providecommand{\tightlist}{%
      \setlength{\itemsep}{0pt}\setlength{\parskip}{0pt}}
    \DefineVerbatimEnvironment{Highlighting}{Verbatim}{commandchars=\\\{\}}
    % Add ',fontsize=\small' for more characters per line
    \newenvironment{Shaded}{}{}
    \newcommand{\KeywordTok}[1]{\textcolor[rgb]{0.00,0.44,0.13}{\textbf{{#1}}}}
    \newcommand{\DataTypeTok}[1]{\textcolor[rgb]{0.56,0.13,0.00}{{#1}}}
    \newcommand{\DecValTok}[1]{\textcolor[rgb]{0.25,0.63,0.44}{{#1}}}
    \newcommand{\BaseNTok}[1]{\textcolor[rgb]{0.25,0.63,0.44}{{#1}}}
    \newcommand{\FloatTok}[1]{\textcolor[rgb]{0.25,0.63,0.44}{{#1}}}
    \newcommand{\CharTok}[1]{\textcolor[rgb]{0.25,0.44,0.63}{{#1}}}
    \newcommand{\StringTok}[1]{\textcolor[rgb]{0.25,0.44,0.63}{{#1}}}
    \newcommand{\CommentTok}[1]{\textcolor[rgb]{0.38,0.63,0.69}{\textit{{#1}}}}
    \newcommand{\OtherTok}[1]{\textcolor[rgb]{0.00,0.44,0.13}{{#1}}}
    \newcommand{\AlertTok}[1]{\textcolor[rgb]{1.00,0.00,0.00}{\textbf{{#1}}}}
    \newcommand{\FunctionTok}[1]{\textcolor[rgb]{0.02,0.16,0.49}{{#1}}}
    \newcommand{\RegionMarkerTok}[1]{{#1}}
    \newcommand{\ErrorTok}[1]{\textcolor[rgb]{1.00,0.00,0.00}{\textbf{{#1}}}}
    \newcommand{\NormalTok}[1]{{#1}}
    
    % Additional commands for more recent versions of Pandoc
    \newcommand{\ConstantTok}[1]{\textcolor[rgb]{0.53,0.00,0.00}{{#1}}}
    \newcommand{\SpecialCharTok}[1]{\textcolor[rgb]{0.25,0.44,0.63}{{#1}}}
    \newcommand{\VerbatimStringTok}[1]{\textcolor[rgb]{0.25,0.44,0.63}{{#1}}}
    \newcommand{\SpecialStringTok}[1]{\textcolor[rgb]{0.73,0.40,0.53}{{#1}}}
    \newcommand{\ImportTok}[1]{{#1}}
    \newcommand{\DocumentationTok}[1]{\textcolor[rgb]{0.73,0.13,0.13}{\textit{{#1}}}}
    \newcommand{\AnnotationTok}[1]{\textcolor[rgb]{0.38,0.63,0.69}{\textbf{\textit{{#1}}}}}
    \newcommand{\CommentVarTok}[1]{\textcolor[rgb]{0.38,0.63,0.69}{\textbf{\textit{{#1}}}}}
    \newcommand{\VariableTok}[1]{\textcolor[rgb]{0.10,0.09,0.49}{{#1}}}
    \newcommand{\ControlFlowTok}[1]{\textcolor[rgb]{0.00,0.44,0.13}{\textbf{{#1}}}}
    \newcommand{\OperatorTok}[1]{\textcolor[rgb]{0.40,0.40,0.40}{{#1}}}
    \newcommand{\BuiltInTok}[1]{{#1}}
    \newcommand{\ExtensionTok}[1]{{#1}}
    \newcommand{\PreprocessorTok}[1]{\textcolor[rgb]{0.74,0.48,0.00}{{#1}}}
    \newcommand{\AttributeTok}[1]{\textcolor[rgb]{0.49,0.56,0.16}{{#1}}}
    \newcommand{\InformationTok}[1]{\textcolor[rgb]{0.38,0.63,0.69}{\textbf{\textit{{#1}}}}}
    \newcommand{\WarningTok}[1]{\textcolor[rgb]{0.38,0.63,0.69}{\textbf{\textit{{#1}}}}}
    
    
    % Define a nice break command that doesn't care if a line doesn't already
    % exist.
    \def\br{\hspace*{\fill} \\* }
    % Math Jax compatability definitions
    \def\gt{>}
    \def\lt{<}
    % Document parameters
    \title{A5}
    
    
    

    % Pygments definitions
    
\makeatletter
\def\PY@reset{\let\PY@it=\relax \let\PY@bf=\relax%
    \let\PY@ul=\relax \let\PY@tc=\relax%
    \let\PY@bc=\relax \let\PY@ff=\relax}
\def\PY@tok#1{\csname PY@tok@#1\endcsname}
\def\PY@toks#1+{\ifx\relax#1\empty\else%
    \PY@tok{#1}\expandafter\PY@toks\fi}
\def\PY@do#1{\PY@bc{\PY@tc{\PY@ul{%
    \PY@it{\PY@bf{\PY@ff{#1}}}}}}}
\def\PY#1#2{\PY@reset\PY@toks#1+\relax+\PY@do{#2}}

\expandafter\def\csname PY@tok@w\endcsname{\def\PY@tc##1{\textcolor[rgb]{0.73,0.73,0.73}{##1}}}
\expandafter\def\csname PY@tok@c\endcsname{\let\PY@it=\textit\def\PY@tc##1{\textcolor[rgb]{0.25,0.50,0.50}{##1}}}
\expandafter\def\csname PY@tok@cp\endcsname{\def\PY@tc##1{\textcolor[rgb]{0.74,0.48,0.00}{##1}}}
\expandafter\def\csname PY@tok@k\endcsname{\let\PY@bf=\textbf\def\PY@tc##1{\textcolor[rgb]{0.00,0.50,0.00}{##1}}}
\expandafter\def\csname PY@tok@kp\endcsname{\def\PY@tc##1{\textcolor[rgb]{0.00,0.50,0.00}{##1}}}
\expandafter\def\csname PY@tok@kt\endcsname{\def\PY@tc##1{\textcolor[rgb]{0.69,0.00,0.25}{##1}}}
\expandafter\def\csname PY@tok@o\endcsname{\def\PY@tc##1{\textcolor[rgb]{0.40,0.40,0.40}{##1}}}
\expandafter\def\csname PY@tok@ow\endcsname{\let\PY@bf=\textbf\def\PY@tc##1{\textcolor[rgb]{0.67,0.13,1.00}{##1}}}
\expandafter\def\csname PY@tok@nb\endcsname{\def\PY@tc##1{\textcolor[rgb]{0.00,0.50,0.00}{##1}}}
\expandafter\def\csname PY@tok@nf\endcsname{\def\PY@tc##1{\textcolor[rgb]{0.00,0.00,1.00}{##1}}}
\expandafter\def\csname PY@tok@nc\endcsname{\let\PY@bf=\textbf\def\PY@tc##1{\textcolor[rgb]{0.00,0.00,1.00}{##1}}}
\expandafter\def\csname PY@tok@nn\endcsname{\let\PY@bf=\textbf\def\PY@tc##1{\textcolor[rgb]{0.00,0.00,1.00}{##1}}}
\expandafter\def\csname PY@tok@ne\endcsname{\let\PY@bf=\textbf\def\PY@tc##1{\textcolor[rgb]{0.82,0.25,0.23}{##1}}}
\expandafter\def\csname PY@tok@nv\endcsname{\def\PY@tc##1{\textcolor[rgb]{0.10,0.09,0.49}{##1}}}
\expandafter\def\csname PY@tok@no\endcsname{\def\PY@tc##1{\textcolor[rgb]{0.53,0.00,0.00}{##1}}}
\expandafter\def\csname PY@tok@nl\endcsname{\def\PY@tc##1{\textcolor[rgb]{0.63,0.63,0.00}{##1}}}
\expandafter\def\csname PY@tok@ni\endcsname{\let\PY@bf=\textbf\def\PY@tc##1{\textcolor[rgb]{0.60,0.60,0.60}{##1}}}
\expandafter\def\csname PY@tok@na\endcsname{\def\PY@tc##1{\textcolor[rgb]{0.49,0.56,0.16}{##1}}}
\expandafter\def\csname PY@tok@nt\endcsname{\let\PY@bf=\textbf\def\PY@tc##1{\textcolor[rgb]{0.00,0.50,0.00}{##1}}}
\expandafter\def\csname PY@tok@nd\endcsname{\def\PY@tc##1{\textcolor[rgb]{0.67,0.13,1.00}{##1}}}
\expandafter\def\csname PY@tok@s\endcsname{\def\PY@tc##1{\textcolor[rgb]{0.73,0.13,0.13}{##1}}}
\expandafter\def\csname PY@tok@sd\endcsname{\let\PY@it=\textit\def\PY@tc##1{\textcolor[rgb]{0.73,0.13,0.13}{##1}}}
\expandafter\def\csname PY@tok@si\endcsname{\let\PY@bf=\textbf\def\PY@tc##1{\textcolor[rgb]{0.73,0.40,0.53}{##1}}}
\expandafter\def\csname PY@tok@se\endcsname{\let\PY@bf=\textbf\def\PY@tc##1{\textcolor[rgb]{0.73,0.40,0.13}{##1}}}
\expandafter\def\csname PY@tok@sr\endcsname{\def\PY@tc##1{\textcolor[rgb]{0.73,0.40,0.53}{##1}}}
\expandafter\def\csname PY@tok@ss\endcsname{\def\PY@tc##1{\textcolor[rgb]{0.10,0.09,0.49}{##1}}}
\expandafter\def\csname PY@tok@sx\endcsname{\def\PY@tc##1{\textcolor[rgb]{0.00,0.50,0.00}{##1}}}
\expandafter\def\csname PY@tok@m\endcsname{\def\PY@tc##1{\textcolor[rgb]{0.40,0.40,0.40}{##1}}}
\expandafter\def\csname PY@tok@gh\endcsname{\let\PY@bf=\textbf\def\PY@tc##1{\textcolor[rgb]{0.00,0.00,0.50}{##1}}}
\expandafter\def\csname PY@tok@gu\endcsname{\let\PY@bf=\textbf\def\PY@tc##1{\textcolor[rgb]{0.50,0.00,0.50}{##1}}}
\expandafter\def\csname PY@tok@gd\endcsname{\def\PY@tc##1{\textcolor[rgb]{0.63,0.00,0.00}{##1}}}
\expandafter\def\csname PY@tok@gi\endcsname{\def\PY@tc##1{\textcolor[rgb]{0.00,0.63,0.00}{##1}}}
\expandafter\def\csname PY@tok@gr\endcsname{\def\PY@tc##1{\textcolor[rgb]{1.00,0.00,0.00}{##1}}}
\expandafter\def\csname PY@tok@ge\endcsname{\let\PY@it=\textit}
\expandafter\def\csname PY@tok@gs\endcsname{\let\PY@bf=\textbf}
\expandafter\def\csname PY@tok@gp\endcsname{\let\PY@bf=\textbf\def\PY@tc##1{\textcolor[rgb]{0.00,0.00,0.50}{##1}}}
\expandafter\def\csname PY@tok@go\endcsname{\def\PY@tc##1{\textcolor[rgb]{0.53,0.53,0.53}{##1}}}
\expandafter\def\csname PY@tok@gt\endcsname{\def\PY@tc##1{\textcolor[rgb]{0.00,0.27,0.87}{##1}}}
\expandafter\def\csname PY@tok@err\endcsname{\def\PY@bc##1{\setlength{\fboxsep}{0pt}\fcolorbox[rgb]{1.00,0.00,0.00}{1,1,1}{\strut ##1}}}
\expandafter\def\csname PY@tok@kc\endcsname{\let\PY@bf=\textbf\def\PY@tc##1{\textcolor[rgb]{0.00,0.50,0.00}{##1}}}
\expandafter\def\csname PY@tok@kd\endcsname{\let\PY@bf=\textbf\def\PY@tc##1{\textcolor[rgb]{0.00,0.50,0.00}{##1}}}
\expandafter\def\csname PY@tok@kn\endcsname{\let\PY@bf=\textbf\def\PY@tc##1{\textcolor[rgb]{0.00,0.50,0.00}{##1}}}
\expandafter\def\csname PY@tok@kr\endcsname{\let\PY@bf=\textbf\def\PY@tc##1{\textcolor[rgb]{0.00,0.50,0.00}{##1}}}
\expandafter\def\csname PY@tok@bp\endcsname{\def\PY@tc##1{\textcolor[rgb]{0.00,0.50,0.00}{##1}}}
\expandafter\def\csname PY@tok@fm\endcsname{\def\PY@tc##1{\textcolor[rgb]{0.00,0.00,1.00}{##1}}}
\expandafter\def\csname PY@tok@vc\endcsname{\def\PY@tc##1{\textcolor[rgb]{0.10,0.09,0.49}{##1}}}
\expandafter\def\csname PY@tok@vg\endcsname{\def\PY@tc##1{\textcolor[rgb]{0.10,0.09,0.49}{##1}}}
\expandafter\def\csname PY@tok@vi\endcsname{\def\PY@tc##1{\textcolor[rgb]{0.10,0.09,0.49}{##1}}}
\expandafter\def\csname PY@tok@vm\endcsname{\def\PY@tc##1{\textcolor[rgb]{0.10,0.09,0.49}{##1}}}
\expandafter\def\csname PY@tok@sa\endcsname{\def\PY@tc##1{\textcolor[rgb]{0.73,0.13,0.13}{##1}}}
\expandafter\def\csname PY@tok@sb\endcsname{\def\PY@tc##1{\textcolor[rgb]{0.73,0.13,0.13}{##1}}}
\expandafter\def\csname PY@tok@sc\endcsname{\def\PY@tc##1{\textcolor[rgb]{0.73,0.13,0.13}{##1}}}
\expandafter\def\csname PY@tok@dl\endcsname{\def\PY@tc##1{\textcolor[rgb]{0.73,0.13,0.13}{##1}}}
\expandafter\def\csname PY@tok@s2\endcsname{\def\PY@tc##1{\textcolor[rgb]{0.73,0.13,0.13}{##1}}}
\expandafter\def\csname PY@tok@sh\endcsname{\def\PY@tc##1{\textcolor[rgb]{0.73,0.13,0.13}{##1}}}
\expandafter\def\csname PY@tok@s1\endcsname{\def\PY@tc##1{\textcolor[rgb]{0.73,0.13,0.13}{##1}}}
\expandafter\def\csname PY@tok@mb\endcsname{\def\PY@tc##1{\textcolor[rgb]{0.40,0.40,0.40}{##1}}}
\expandafter\def\csname PY@tok@mf\endcsname{\def\PY@tc##1{\textcolor[rgb]{0.40,0.40,0.40}{##1}}}
\expandafter\def\csname PY@tok@mh\endcsname{\def\PY@tc##1{\textcolor[rgb]{0.40,0.40,0.40}{##1}}}
\expandafter\def\csname PY@tok@mi\endcsname{\def\PY@tc##1{\textcolor[rgb]{0.40,0.40,0.40}{##1}}}
\expandafter\def\csname PY@tok@il\endcsname{\def\PY@tc##1{\textcolor[rgb]{0.40,0.40,0.40}{##1}}}
\expandafter\def\csname PY@tok@mo\endcsname{\def\PY@tc##1{\textcolor[rgb]{0.40,0.40,0.40}{##1}}}
\expandafter\def\csname PY@tok@ch\endcsname{\let\PY@it=\textit\def\PY@tc##1{\textcolor[rgb]{0.25,0.50,0.50}{##1}}}
\expandafter\def\csname PY@tok@cm\endcsname{\let\PY@it=\textit\def\PY@tc##1{\textcolor[rgb]{0.25,0.50,0.50}{##1}}}
\expandafter\def\csname PY@tok@cpf\endcsname{\let\PY@it=\textit\def\PY@tc##1{\textcolor[rgb]{0.25,0.50,0.50}{##1}}}
\expandafter\def\csname PY@tok@c1\endcsname{\let\PY@it=\textit\def\PY@tc##1{\textcolor[rgb]{0.25,0.50,0.50}{##1}}}
\expandafter\def\csname PY@tok@cs\endcsname{\let\PY@it=\textit\def\PY@tc##1{\textcolor[rgb]{0.25,0.50,0.50}{##1}}}

\def\PYZbs{\char`\\}
\def\PYZus{\char`\_}
\def\PYZob{\char`\{}
\def\PYZcb{\char`\}}
\def\PYZca{\char`\^}
\def\PYZam{\char`\&}
\def\PYZlt{\char`\<}
\def\PYZgt{\char`\>}
\def\PYZsh{\char`\#}
\def\PYZpc{\char`\%}
\def\PYZdl{\char`\$}
\def\PYZhy{\char`\-}
\def\PYZsq{\char`\'}
\def\PYZdq{\char`\"}
\def\PYZti{\char`\~}
% for compatibility with earlier versions
\def\PYZat{@}
\def\PYZlb{[}
\def\PYZrb{]}
\makeatother


    % Exact colors from NB
    \definecolor{incolor}{rgb}{0.0, 0.0, 0.5}
    \definecolor{outcolor}{rgb}{0.545, 0.0, 0.0}



    
    % Prevent overflowing lines due to hard-to-break entities
    \sloppy 
    % Setup hyperref package
    \hypersetup{
      breaklinks=true,  % so long urls are correctly broken across lines
      colorlinks=true,
      urlcolor=urlcolor,
      linkcolor=linkcolor,
      citecolor=citecolor,
      }
    % Slightly bigger margins than the latex defaults
    
    \geometry{verbose,tmargin=1in,bmargin=1in,lmargin=1in,rmargin=1in}
    
    

    \begin{document}
    
    
    \maketitle
    
    

    
    \hypertarget{eecs531---a5}{%
\section{EECS531 - A5}\label{eecs531---a5}}

\hypertarget{tristan-maidment-tdm47}{%
\paragraph{Tristan Maidment {[}tdm47{]}}\label{tristan-maidment-tdm47}}

\hypertarget{goal}{%
\paragraph{Goal}\label{goal}}

The purpose of this assignment is to comment all of the code provided to
demonstrate our understanding of the material. In order to better
understand the functionality, I re-implemented the logic in
numpy/python.

\hypertarget{implementationexplanation}{%
\subsubsection{Implementation/Explanation}\label{implementationexplanation}}

    This function is used to define the object viewed by the cameras. A cube
represented by 9 points are defined in space. The simulation coordinates
are with respect to the center of projection. This means that the center
of the cube is located at the center of projection, at location
\texttt{{[}0,\ 0,\ 0{]}}.

In addition, it provides a color map for easy point identification.

    \begin{Verbatim}[commandchars=\\\{\}]
{\color{incolor}In [{\color{incolor}52}]:} \PY{o}{\PYZpc{}}\PY{k}{matplotlib} inline
         \PY{k+kn}{import} \PY{n+nn}{numpy} \PY{k}{as} \PY{n+nn}{np}
         \PY{k+kn}{import} \PY{n+nn}{matplotlib}\PY{n+nn}{.}\PY{n+nn}{pyplot} \PY{k}{as} \PY{n+nn}{plt}
         
         \PY{k}{def} \PY{n+nf}{create\PYZus{}points}\PY{p}{(}\PY{p}{)}\PY{p}{:}
             \PY{n}{X}\PY{p}{,} \PY{n}{Y}\PY{p}{,} \PY{n}{Z} \PY{o}{=} \PY{n}{np}\PY{o}{.}\PY{n}{meshgrid}\PY{p}{(}\PY{p}{[}\PY{o}{\PYZhy{}}\PY{l+m+mf}{0.5}\PY{p}{,} \PY{l+m+mi}{0}\PY{p}{,} \PY{l+m+mf}{0.5}\PY{p}{]}\PY{p}{,} \PY{p}{[}\PY{o}{\PYZhy{}}\PY{l+m+mf}{0.5}\PY{p}{,} \PY{l+m+mi}{0}\PY{p}{,} \PY{l+m+mf}{0.5}\PY{p}{]}\PY{p}{,} \PY{p}{[}\PY{o}{\PYZhy{}}\PY{l+m+mf}{0.5}\PY{p}{,} \PY{l+m+mi}{0}\PY{p}{,} \PY{l+m+mf}{0.5}\PY{p}{]}\PY{p}{)}
             \PY{n}{points} \PY{o}{=} \PY{n}{np}\PY{o}{.}\PY{n}{vstack}\PY{p}{(}\PY{p}{[}\PY{n}{np}\PY{o}{.}\PY{n}{ravel}\PY{p}{(}\PY{n}{X}\PY{p}{)}\PY{p}{,} \PY{n}{np}\PY{o}{.}\PY{n}{ravel}\PY{p}{(}\PY{n}{Y}\PY{p}{)}\PY{p}{,} \PY{n}{np}\PY{o}{.}\PY{n}{ravel}\PY{p}{(}\PY{n}{Z}\PY{p}{)}\PY{p}{]}\PY{p}{)}
             \PY{k}{return} \PY{n}{points}\PY{p}{,} \PY{n}{plt}\PY{o}{.}\PY{n}{cm}\PY{o}{.}\PY{n}{prism}\PY{p}{(}\PY{n+nb}{range}\PY{p}{(}\PY{n}{points}\PY{p}{[}\PY{l+m+mi}{0}\PY{p}{]}\PY{o}{.}\PY{n}{shape}\PY{p}{[}\PY{l+m+mi}{0}\PY{p}{]}\PY{p}{)}\PY{p}{)}
         
         \PY{n}{points}\PY{p}{,} \PY{n}{colors} \PY{o}{=} \PY{n}{create\PYZus{}points}\PY{p}{(}\PY{p}{)}\PY{p}{;}
         \PY{n}{fig} \PY{o}{=} \PY{n}{plt}\PY{o}{.}\PY{n}{figure}\PY{p}{(}\PY{n}{figsize}\PY{o}{=}\PY{p}{(}\PY{l+m+mi}{8}\PY{p}{,} \PY{l+m+mi}{6}\PY{p}{)}\PY{p}{,} \PY{n}{dpi}\PY{o}{=}\PY{l+m+mi}{160}\PY{p}{)}
         \PY{n}{ax} \PY{o}{=} \PY{n}{fig}\PY{o}{.}\PY{n}{gca}\PY{p}{(}\PY{n}{projection}\PY{o}{=}\PY{l+s+s1}{\PYZsq{}}\PY{l+s+s1}{3d}\PY{l+s+s1}{\PYZsq{}}\PY{p}{)}
         \PY{n}{ax}\PY{o}{.}\PY{n}{scatter}\PY{p}{(}\PY{n}{points}\PY{p}{[}\PY{l+m+mi}{0}\PY{p}{]}\PY{p}{,} \PY{n}{points}\PY{p}{[}\PY{l+m+mi}{1}\PY{p}{]}\PY{p}{,} \PY{n}{points}\PY{p}{[}\PY{l+m+mi}{2}\PY{p}{]}\PY{p}{,} \PY{n}{c}\PY{o}{=}\PY{n}{colors}\PY{p}{,} \PY{n}{marker}\PY{o}{=}\PY{l+s+s2}{\PYZdq{}}\PY{l+s+s2}{o}\PY{l+s+s2}{\PYZdq{}}\PY{p}{)}
         \PY{n}{plt}\PY{o}{.}\PY{n}{show}\PY{p}{(}\PY{p}{)}
\end{Verbatim}


    \begin{center}
    \adjustimage{max size={0.9\linewidth}{0.9\paperheight}}{output_2_0.png}
    \end{center}
    { \hspace*{\fill} \\}
    
    In addition, both cameras must be defined.

\textbf{Position} - Since the coordinate system being used is with
respect to the center of projection, the position of the cameras is
specified with a distance of 5 units from the object.

\textbf{Target} - This defines the focal point of the camera. Since the
camera is focusing on the cube defined at \texttt{{[}0,\ 0,\ 0{]}}, the
focal point is also located here.

\textbf{Up} - This metric is used to define the ``up'' vector,
specifically for when converting to the camera coordinate system.

\textbf{Focal\_Length} - Specifies the distance between the lens and
detector.

\textbf{Film\_Width/Film\_Height} - These values represent the size of
the dimensions of the detector element within the camera.

\textbf{Width/Height} - The width and height represents the size (in
pixels) of the plane being at the focal point. The ratio between
\texttt{height} and \texttt{width} is equal to that of the
\texttt{film\_width} and \texttt{film\_height}.

    \begin{Verbatim}[commandchars=\\\{\}]
{\color{incolor}In [{\color{incolor}2}]:} \PY{k}{def} \PY{n+nf}{preset\PYZus{}cameras}\PY{p}{(}\PY{p}{)}\PY{p}{:}
            \PY{n}{r} \PY{o}{=} \PY{l+m+mi}{5}
            \PY{n}{alpha} \PY{o}{=} \PY{n}{np}\PY{o}{.}\PY{n}{pi}\PY{o}{/}\PY{l+m+mi}{6}
            \PY{n}{beta} \PY{o}{=} \PY{n}{np}\PY{o}{.}\PY{n}{pi}\PY{o}{/}\PY{l+m+mi}{6}
            \PY{n}{cam1} \PY{o}{=} \PY{p}{\PYZob{}}\PY{l+s+s1}{\PYZsq{}}\PY{l+s+s1}{position}\PY{l+s+s1}{\PYZsq{}}\PY{p}{:} \PY{n}{np}\PY{o}{.}\PY{n}{asarray}\PY{p}{(}\PY{p}{[}\PY{n}{r}\PY{o}{*}\PY{n}{np}\PY{o}{.}\PY{n}{cos}\PY{p}{(}\PY{n}{beta}\PY{p}{)}\PY{o}{*}\PY{n}{np}\PY{o}{.}\PY{n}{cos}\PY{p}{(}\PY{n}{alpha}\PY{p}{)}\PY{p}{,} \PY{n}{r}\PY{o}{*}\PY{n}{np}\PY{o}{.}\PY{n}{cos}\PY{p}{(}\PY{n}{beta}\PY{p}{)}\PY{o}{*}\PY{n}{np}\PY{o}{.}\PY{n}{sin}\PY{p}{(}\PY{n}{alpha}\PY{p}{)}\PY{p}{,} \PY{n}{r}\PY{o}{*}\PY{n}{np}\PY{o}{.}\PY{n}{sin}\PY{p}{(}\PY{n}{beta}\PY{p}{)}\PY{p}{]}\PY{p}{)}\PY{p}{,}
                    \PY{l+s+s1}{\PYZsq{}}\PY{l+s+s1}{target}\PY{l+s+s1}{\PYZsq{}}\PY{p}{:} \PY{n}{np}\PY{o}{.}\PY{n}{asarray}\PY{p}{(}\PY{p}{[}\PY{l+m+mi}{0}\PY{p}{,} \PY{l+m+mi}{0}\PY{p}{,} \PY{l+m+mi}{0}\PY{p}{]}\PY{p}{)}\PY{p}{,}
                    \PY{l+s+s1}{\PYZsq{}}\PY{l+s+s1}{up}\PY{l+s+s1}{\PYZsq{}}\PY{p}{:} \PY{n}{np}\PY{o}{.}\PY{n}{asarray}\PY{p}{(}\PY{p}{[}\PY{l+m+mi}{0}\PY{p}{,} \PY{l+m+mi}{0}\PY{p}{,} \PY{l+m+mi}{1}\PY{p}{]}\PY{p}{)}\PY{p}{,}
                    \PY{l+s+s1}{\PYZsq{}}\PY{l+s+s1}{focal\PYZus{}length}\PY{l+s+s1}{\PYZsq{}}\PY{p}{:} \PY{l+m+mf}{0.06}\PY{p}{,} 
                    \PY{l+s+s1}{\PYZsq{}}\PY{l+s+s1}{film\PYZus{}width}\PY{l+s+s1}{\PYZsq{}}\PY{p}{:} \PY{l+m+mf}{0.035}\PY{p}{,}
                    \PY{l+s+s1}{\PYZsq{}}\PY{l+s+s1}{film\PYZus{}height}\PY{l+s+s1}{\PYZsq{}}\PY{p}{:} \PY{l+m+mf}{0.035}\PY{p}{,}
                    \PY{l+s+s1}{\PYZsq{}}\PY{l+s+s1}{width}\PY{l+s+s1}{\PYZsq{}}\PY{p}{:} \PY{l+m+mi}{256}\PY{p}{,}
                    \PY{l+s+s1}{\PYZsq{}}\PY{l+s+s1}{height}\PY{l+s+s1}{\PYZsq{}}\PY{p}{:} \PY{l+m+mi}{256}\PY{p}{\PYZcb{}}
            \PY{n}{alpha} \PY{o}{=} \PY{n}{np}\PY{o}{.}\PY{n}{pi}\PY{o}{/}\PY{l+m+mi}{3}
            \PY{n}{cam2} \PY{o}{=} \PY{p}{\PYZob{}}\PY{l+s+s1}{\PYZsq{}}\PY{l+s+s1}{position}\PY{l+s+s1}{\PYZsq{}}\PY{p}{:} \PY{n}{np}\PY{o}{.}\PY{n}{asarray}\PY{p}{(}\PY{p}{[}\PY{n}{r}\PY{o}{*}\PY{n}{np}\PY{o}{.}\PY{n}{cos}\PY{p}{(}\PY{n}{beta}\PY{p}{)}\PY{o}{*}\PY{n}{np}\PY{o}{.}\PY{n}{cos}\PY{p}{(}\PY{n}{alpha}\PY{p}{)}\PY{p}{,} \PY{n}{r}\PY{o}{*}\PY{n}{np}\PY{o}{.}\PY{n}{cos}\PY{p}{(}\PY{n}{beta}\PY{p}{)}\PY{o}{*}\PY{n}{np}\PY{o}{.}\PY{n}{sin}\PY{p}{(}\PY{n}{alpha}\PY{p}{)}\PY{p}{,} \PY{n}{r}\PY{o}{*}\PY{n}{np}\PY{o}{.}\PY{n}{sin}\PY{p}{(}\PY{n}{beta}\PY{p}{)}\PY{p}{]}\PY{p}{)}\PY{p}{,}
                    \PY{l+s+s1}{\PYZsq{}}\PY{l+s+s1}{target}\PY{l+s+s1}{\PYZsq{}}\PY{p}{:} \PY{n}{np}\PY{o}{.}\PY{n}{asarray}\PY{p}{(}\PY{p}{[}\PY{l+m+mi}{0}\PY{p}{,} \PY{l+m+mi}{0}\PY{p}{,} \PY{l+m+mi}{0}\PY{p}{]}\PY{p}{)}\PY{p}{,}
                    \PY{l+s+s1}{\PYZsq{}}\PY{l+s+s1}{up}\PY{l+s+s1}{\PYZsq{}}\PY{p}{:} \PY{n}{np}\PY{o}{.}\PY{n}{asarray}\PY{p}{(}\PY{p}{[}\PY{l+m+mi}{0}\PY{p}{,} \PY{l+m+mi}{0}\PY{p}{,} \PY{l+m+mi}{1}\PY{p}{]}\PY{p}{)}\PY{p}{,}
                    \PY{l+s+s1}{\PYZsq{}}\PY{l+s+s1}{focal\PYZus{}length}\PY{l+s+s1}{\PYZsq{}}\PY{p}{:} \PY{l+m+mf}{0.06}\PY{p}{,} 
                    \PY{l+s+s1}{\PYZsq{}}\PY{l+s+s1}{film\PYZus{}width}\PY{l+s+s1}{\PYZsq{}}\PY{p}{:} \PY{l+m+mf}{0.035}\PY{p}{,}
                    \PY{l+s+s1}{\PYZsq{}}\PY{l+s+s1}{film\PYZus{}height}\PY{l+s+s1}{\PYZsq{}}\PY{p}{:} \PY{l+m+mf}{0.035}\PY{p}{,}
                    \PY{l+s+s1}{\PYZsq{}}\PY{l+s+s1}{width}\PY{l+s+s1}{\PYZsq{}}\PY{p}{:} \PY{l+m+mi}{256}\PY{p}{,}
                    \PY{l+s+s1}{\PYZsq{}}\PY{l+s+s1}{height}\PY{l+s+s1}{\PYZsq{}}\PY{p}{:} \PY{l+m+mi}{256}\PY{p}{\PYZcb{}}
            \PY{k}{return} \PY{n}{cam1}\PY{p}{,} \PY{n}{cam2}
\end{Verbatim}


    \textbf{zcam} - This function determines the coordinate system of the
camera by first finding the vector between the target and the camera
origin, labeled \texttt{zcam}. This axis goes from the target to the
origin of the camera, and represents depth.

\textbf{xcam} - \texttt{xcam} is determined by taking the cross product
of zcam and the ``up'' vector, which determines the horizontal axis of
the camera.

\textbf{ycam} - \texttt{ycam} uses this horizontal axis, \texttt{xcam},
and the distance axis, \texttt{zcam}, to determine the vertical axis.

    \begin{Verbatim}[commandchars=\\\{\}]
{\color{incolor}In [{\color{incolor}3}]:} \PY{k}{def} \PY{n+nf}{camera\PYZus{}coordinate\PYZus{}system}\PY{p}{(}\PY{n}{cam}\PY{p}{)}\PY{p}{:}
            \PY{c+c1}{\PYZsh{} determine camera distance from target}
            \PY{n}{zcam} \PY{o}{=} \PY{n}{cam}\PY{p}{[}\PY{l+s+s1}{\PYZsq{}}\PY{l+s+s1}{target}\PY{l+s+s1}{\PYZsq{}}\PY{p}{]} \PY{o}{\PYZhy{}} \PY{n}{cam}\PY{p}{[}\PY{l+s+s1}{\PYZsq{}}\PY{l+s+s1}{position}\PY{l+s+s1}{\PYZsq{}}\PY{p}{]}
            \PY{n}{xcam} \PY{o}{=} \PY{n}{np}\PY{o}{.}\PY{n}{cross}\PY{p}{(}\PY{n}{zcam}\PY{p}{,} \PY{n}{cam}\PY{p}{[}\PY{l+s+s1}{\PYZsq{}}\PY{l+s+s1}{up}\PY{l+s+s1}{\PYZsq{}}\PY{p}{]}\PY{p}{)}
            \PY{n}{ycam} \PY{o}{=} \PY{n}{np}\PY{o}{.}\PY{n}{cross}\PY{p}{(}\PY{n}{zcam}\PY{p}{,} \PY{n}{xcam}\PY{p}{)}
            
            \PY{n}{zcam} \PY{o}{/}\PY{o}{=} \PY{n}{np}\PY{o}{.}\PY{n}{linalg}\PY{o}{.}\PY{n}{norm}\PY{p}{(}\PY{n}{zcam}\PY{p}{)}
            \PY{n}{xcam} \PY{o}{/}\PY{o}{=} \PY{n}{np}\PY{o}{.}\PY{n}{linalg}\PY{o}{.}\PY{n}{norm}\PY{p}{(}\PY{n}{xcam}\PY{p}{)}
            \PY{n}{ycam} \PY{o}{/}\PY{o}{=} \PY{n}{np}\PY{o}{.}\PY{n}{linalg}\PY{o}{.}\PY{n}{norm}\PY{p}{(}\PY{n}{ycam}\PY{p}{)}
            
            \PY{n}{origin} \PY{o}{=} \PY{n}{cam}\PY{p}{[}\PY{l+s+s1}{\PYZsq{}}\PY{l+s+s1}{position}\PY{l+s+s1}{\PYZsq{}}\PY{p}{]}
            \PY{k}{return} \PY{n}{xcam}\PY{p}{,} \PY{n}{ycam}\PY{p}{,} \PY{n}{zcam}\PY{p}{,} \PY{n}{origin}
\end{Verbatim}


    \texttt{plot\_camera} draws the camera acquisition geometry in the
object space, to demonstrate the positioning of each camera. Using the
camera coordinate system, \texttt{plot\_camera} uses both the
\texttt{film\_width} and \texttt{focal\_length} to determine the four
corner points of the camera acqusition plane. The camera is drawn with
the ``origin'' being the location of the lens.

To calculate the location of the four corners of the acquisiton plane,
the location of the four corners of the detector determined first. A
line can be drawn from the four corners of the detector to the opposite
corners in the acquisition plane, through the center of the lens. Since
the center of the detector is in line with the center of the lense, only
two values, \texttt{x} and \texttt{y}, can represent the the distance
from the center in the X and Y axis. By varying the sign of \texttt{x}
and \texttt{y}, we can represent all four points \texttt{(x,y)},
\texttt{(-x,y)}, \texttt{(x,-y)}, \texttt{(-x,-y)}. Distance \texttt{d}
between the target and the camera, and is divided by the
\texttt{focal\_length} to properly scale \texttt{x} and \texttt{y} from
detector size to acquisition plane size.

After calculating \texttt{x}, \texttt{y}, and \texttt{d}, the axis of
the acquisition plane can be determined. The external geometry is
exactly the same as the internal geometry, but flipped.

After determining the four points in the acquisition plane, the plane is
rendered in 3D, and lines are drawn from the lense to the points, to
visualize the projection.

    \begin{Verbatim}[commandchars=\\\{\}]
{\color{incolor}In [{\color{incolor}120}]:} \PY{k+kn}{from} \PY{n+nn}{mpl\PYZus{}toolkits}\PY{n+nn}{.}\PY{n+nn}{mplot3d} \PY{k}{import} \PY{n}{Axes3D}
          \PY{k+kn}{from} \PY{n+nn}{mpl\PYZus{}toolkits}\PY{n+nn}{.}\PY{n+nn}{mplot3d} \PY{k}{import} \PY{n}{art3d}
          
          \PY{k}{def} \PY{n+nf}{plot\PYZus{}camera}\PY{p}{(}\PY{n}{cam}\PY{p}{,} \PY{n}{label}\PY{p}{,} \PY{n}{color}\PY{p}{)}\PY{p}{:}
              
              \PY{n}{xcam}\PY{p}{,} \PY{n}{ycam}\PY{p}{,} \PY{n}{zcam}\PY{p}{,} \PY{n}{origin} \PY{o}{=} \PY{n}{camera\PYZus{}coordinate\PYZus{}system}\PY{p}{(}\PY{n}{cam}\PY{p}{)}
              
              \PY{n}{d} \PY{o}{=} \PY{n}{np}\PY{o}{.}\PY{n}{linalg}\PY{o}{.}\PY{n}{norm}\PY{p}{(}\PY{n}{cam}\PY{p}{[}\PY{l+s+s1}{\PYZsq{}}\PY{l+s+s1}{target}\PY{l+s+s1}{\PYZsq{}}\PY{p}{]} \PY{o}{\PYZhy{}} \PY{n}{cam}\PY{p}{[}\PY{l+s+s1}{\PYZsq{}}\PY{l+s+s1}{position}\PY{l+s+s1}{\PYZsq{}}\PY{p}{]}\PY{p}{)}
              \PY{n}{x} \PY{o}{=} \PY{l+m+mf}{0.5} \PY{o}{*} \PY{n}{cam}\PY{p}{[}\PY{l+s+s1}{\PYZsq{}}\PY{l+s+s1}{film\PYZus{}width}\PY{l+s+s1}{\PYZsq{}}\PY{p}{]} \PY{o}{*} \PY{n}{d} \PY{o}{/} \PY{n}{cam}\PY{p}{[}\PY{l+s+s1}{\PYZsq{}}\PY{l+s+s1}{focal\PYZus{}length}\PY{l+s+s1}{\PYZsq{}}\PY{p}{]}
              \PY{n}{y} \PY{o}{=} \PY{l+m+mf}{0.5} \PY{o}{*} \PY{n}{cam}\PY{p}{[}\PY{l+s+s1}{\PYZsq{}}\PY{l+s+s1}{film\PYZus{}width}\PY{l+s+s1}{\PYZsq{}}\PY{p}{]} \PY{o}{*} \PY{n}{d} \PY{o}{/} \PY{n}{cam}\PY{p}{[}\PY{l+s+s1}{\PYZsq{}}\PY{l+s+s1}{focal\PYZus{}length}\PY{l+s+s1}{\PYZsq{}}\PY{p}{]}
              
              \PY{c+c1}{\PYZsh{}label the camera}
              \PY{n}{ax}\PY{o}{.}\PY{n}{text}\PY{p}{(}\PY{n}{origin}\PY{p}{[}\PY{l+m+mi}{0}\PY{p}{]}\PY{p}{,} \PY{n}{origin}\PY{p}{[}\PY{l+m+mi}{1}\PY{p}{]}\PY{p}{,} \PY{n}{origin}\PY{p}{[}\PY{l+m+mi}{2}\PY{p}{]}\PY{p}{,} \PY{n}{label}\PY{p}{)}
              
              \PY{c+c1}{\PYZsh{} determine the four points on the acquisition plane corners}
              \PY{n}{P1} \PY{o}{=} \PY{n}{origin} \PY{o}{+} \PY{n}{x} \PY{o}{*} \PY{n}{xcam} \PY{o}{+} \PY{n}{y} \PY{o}{*} \PY{n}{ycam} \PY{o}{+} \PY{n}{d} \PY{o}{*} \PY{n}{zcam}
              \PY{n}{P2} \PY{o}{=} \PY{n}{origin} \PY{o}{+} \PY{n}{x} \PY{o}{*} \PY{n}{xcam} \PY{o}{\PYZhy{}} \PY{n}{y} \PY{o}{*} \PY{n}{ycam} \PY{o}{+} \PY{n}{d} \PY{o}{*} \PY{n}{zcam}
              \PY{n}{P3} \PY{o}{=} \PY{n}{origin} \PY{o}{\PYZhy{}} \PY{n}{x} \PY{o}{*} \PY{n}{xcam} \PY{o}{\PYZhy{}} \PY{n}{y} \PY{o}{*} \PY{n}{ycam} \PY{o}{+} \PY{n}{d} \PY{o}{*} \PY{n}{zcam}
              \PY{n}{P4} \PY{o}{=} \PY{n}{origin} \PY{o}{\PYZhy{}} \PY{n}{x} \PY{o}{*} \PY{n}{xcam} \PY{o}{+} \PY{n}{y} \PY{o}{*} \PY{n}{ycam} \PY{o}{+} \PY{n}{d} \PY{o}{*} \PY{n}{zcam}
              
              \PY{n}{connect} \PY{o}{=} \PY{k}{lambda} \PY{n}{p1}\PY{p}{,} \PY{n}{p2}\PY{p}{:} \PY{n}{ax}\PY{o}{.}\PY{n}{plot}\PY{p}{(}\PY{p}{[}\PY{n}{p1}\PY{p}{[}\PY{l+m+mi}{0}\PY{p}{]}\PY{p}{,} \PY{n}{p2}\PY{p}{[}\PY{l+m+mi}{0}\PY{p}{]}\PY{p}{]}\PY{p}{,} \PY{p}{[}\PY{n}{p1}\PY{p}{[}\PY{l+m+mi}{1}\PY{p}{]}\PY{p}{,} \PY{n}{p2}\PY{p}{[}\PY{l+m+mi}{1}\PY{p}{]}\PY{p}{]}\PY{p}{,} \PY{p}{[}\PY{n}{p1}\PY{p}{[}\PY{l+m+mi}{2}\PY{p}{]}\PY{p}{,} \PY{n}{p2}\PY{p}{[}\PY{l+m+mi}{2}\PY{p}{]}\PY{p}{]}\PY{p}{,} \PY{n}{c}\PY{o}{=}\PY{n}{color}\PY{p}{)}
              
              \PY{n}{plane} \PY{o}{=} \PY{n}{np}\PY{o}{.}\PY{n}{stack}\PY{p}{(}\PY{p}{[}\PY{n}{P1}\PY{p}{,} \PY{n}{P2}\PY{p}{,} \PY{n}{P3}\PY{p}{,} \PY{n}{P4}\PY{p}{]}\PY{p}{)}
              \PY{n}{ax}\PY{o}{.}\PY{n}{plot\PYZus{}trisurf}\PY{p}{(}\PY{n}{plane}\PY{p}{[}\PY{p}{:}\PY{p}{,}\PY{l+m+mi}{0}\PY{p}{]}\PY{p}{,} \PY{n}{plane}\PY{p}{[}\PY{p}{:}\PY{p}{,}\PY{l+m+mi}{1}\PY{p}{]}\PY{p}{,} \PY{n}{plane}\PY{p}{[}\PY{p}{:}\PY{p}{,}\PY{l+m+mi}{2}\PY{p}{]}\PY{p}{,} \PY{n}{color}\PY{o}{=}\PY{n}{color}\PY{p}{)}
              
              \PY{n}{connect}\PY{p}{(}\PY{n}{cam}\PY{p}{[}\PY{l+s+s1}{\PYZsq{}}\PY{l+s+s1}{position}\PY{l+s+s1}{\PYZsq{}}\PY{p}{]}\PY{p}{,} \PY{n}{cam}\PY{p}{[}\PY{l+s+s1}{\PYZsq{}}\PY{l+s+s1}{target}\PY{l+s+s1}{\PYZsq{}}\PY{p}{]}\PY{p}{)}
              \PY{n}{connect}\PY{p}{(}\PY{n}{cam}\PY{p}{[}\PY{l+s+s1}{\PYZsq{}}\PY{l+s+s1}{position}\PY{l+s+s1}{\PYZsq{}}\PY{p}{]}\PY{p}{,} \PY{n}{P1}\PY{p}{)}
              \PY{n}{connect}\PY{p}{(}\PY{n}{cam}\PY{p}{[}\PY{l+s+s1}{\PYZsq{}}\PY{l+s+s1}{position}\PY{l+s+s1}{\PYZsq{}}\PY{p}{]}\PY{p}{,} \PY{n}{P2}\PY{p}{)}
              \PY{n}{connect}\PY{p}{(}\PY{n}{cam}\PY{p}{[}\PY{l+s+s1}{\PYZsq{}}\PY{l+s+s1}{position}\PY{l+s+s1}{\PYZsq{}}\PY{p}{]}\PY{p}{,} \PY{n}{P3}\PY{p}{)}
              \PY{n}{connect}\PY{p}{(}\PY{n}{cam}\PY{p}{[}\PY{l+s+s1}{\PYZsq{}}\PY{l+s+s1}{position}\PY{l+s+s1}{\PYZsq{}}\PY{p}{]}\PY{p}{,} \PY{n}{P4}\PY{p}{)}
\end{Verbatim}


    \begin{Verbatim}[commandchars=\\\{\}]
{\color{incolor}In [{\color{incolor}122}]:} \PY{n}{points}\PY{p}{,} \PY{n}{colors} \PY{o}{=} \PY{n}{create\PYZus{}points}\PY{p}{(}\PY{p}{)}\PY{p}{;}
          \PY{n}{cam1}\PY{p}{,} \PY{n}{cam2} \PY{o}{=} \PY{n}{preset\PYZus{}cameras}\PY{p}{(}\PY{p}{)}\PY{p}{;}
\end{Verbatim}


    To visualize the virtual world, a single camera is drawn, with respect
to the object.

    \begin{Verbatim}[commandchars=\\\{\}]
{\color{incolor}In [{\color{incolor}117}]:} \PY{n}{fig} \PY{o}{=} \PY{n}{plt}\PY{o}{.}\PY{n}{figure}\PY{p}{(}\PY{n}{figsize}\PY{o}{=}\PY{p}{(}\PY{l+m+mi}{8}\PY{p}{,} \PY{l+m+mi}{6}\PY{p}{)}\PY{p}{,} \PY{n}{dpi}\PY{o}{=}\PY{l+m+mi}{160}\PY{p}{)}
          \PY{n}{ax} \PY{o}{=} \PY{n}{fig}\PY{o}{.}\PY{n}{gca}\PY{p}{(}\PY{n}{projection}\PY{o}{=}\PY{l+s+s1}{\PYZsq{}}\PY{l+s+s1}{3d}\PY{l+s+s1}{\PYZsq{}}\PY{p}{)}
          \PY{n}{plot\PYZus{}camera}\PY{p}{(}\PY{n}{cam1}\PY{p}{,} \PY{l+s+s1}{\PYZsq{}}\PY{l+s+s1}{Cam 1}\PY{l+s+s1}{\PYZsq{}}\PY{p}{,} \PY{n}{np}\PY{o}{.}\PY{n}{asarray}\PY{p}{(}\PY{p}{[}\PY{l+m+mi}{1}\PY{p}{,} \PY{l+m+mi}{0}\PY{p}{,} \PY{l+m+mi}{0}\PY{p}{,} \PY{l+m+mf}{0.3}\PY{p}{]}\PY{p}{)}\PY{p}{)}
          \PY{n}{ax}\PY{o}{.}\PY{n}{scatter}\PY{p}{(}\PY{n}{points}\PY{p}{[}\PY{l+m+mi}{0}\PY{p}{]}\PY{p}{,} \PY{n}{points}\PY{p}{[}\PY{l+m+mi}{1}\PY{p}{]}\PY{p}{,} \PY{n}{points}\PY{p}{[}\PY{l+m+mi}{2}\PY{p}{]}\PY{p}{,} \PY{n}{c}\PY{o}{=}\PY{n}{colors}\PY{p}{)} 
          \PY{n}{plt}\PY{o}{.}\PY{n}{show}\PY{p}{(}\PY{p}{)}
\end{Verbatim}


    \begin{center}
    \adjustimage{max size={0.9\linewidth}{0.9\paperheight}}{output_11_0.png}
    \end{center}
    { \hspace*{\fill} \\}
    
    Triangulation requires two cameras to work properly. Acquiring 3D
structure can only be done reliably with two cameras, with a decent
amount of distance between them. In addition, they must be angled so
that the focal point is at the same spot. This models what our eyes do
naturally.

    \begin{Verbatim}[commandchars=\\\{\}]
{\color{incolor}In [{\color{incolor}118}]:} \PY{n}{fig} \PY{o}{=} \PY{n}{plt}\PY{o}{.}\PY{n}{figure}\PY{p}{(}\PY{n}{figsize}\PY{o}{=}\PY{p}{(}\PY{l+m+mi}{8}\PY{p}{,} \PY{l+m+mi}{6}\PY{p}{)}\PY{p}{,} \PY{n}{dpi}\PY{o}{=}\PY{l+m+mi}{160}\PY{p}{)}
          \PY{n}{ax} \PY{o}{=} \PY{n}{fig}\PY{o}{.}\PY{n}{gca}\PY{p}{(}\PY{n}{projection}\PY{o}{=}\PY{l+s+s1}{\PYZsq{}}\PY{l+s+s1}{3d}\PY{l+s+s1}{\PYZsq{}}\PY{p}{)}
          \PY{n}{plot\PYZus{}camera}\PY{p}{(}\PY{n}{cam1}\PY{p}{,} \PY{l+s+s1}{\PYZsq{}}\PY{l+s+s1}{Cam 1}\PY{l+s+s1}{\PYZsq{}}\PY{p}{,} \PY{n}{np}\PY{o}{.}\PY{n}{asarray}\PY{p}{(}\PY{p}{[}\PY{l+m+mi}{1}\PY{p}{,} \PY{l+m+mi}{0}\PY{p}{,} \PY{l+m+mi}{0}\PY{p}{,} \PY{l+m+mf}{0.3}\PY{p}{]}\PY{p}{)}\PY{p}{)}
          \PY{n}{plot\PYZus{}camera}\PY{p}{(}\PY{n}{cam2}\PY{p}{,} \PY{l+s+s1}{\PYZsq{}}\PY{l+s+s1}{Cam 2}\PY{l+s+s1}{\PYZsq{}}\PY{p}{,} \PY{n}{np}\PY{o}{.}\PY{n}{asarray}\PY{p}{(}\PY{p}{[}\PY{l+m+mi}{0}\PY{p}{,} \PY{l+m+mi}{0}\PY{p}{,} \PY{l+m+mi}{1}\PY{p}{,} \PY{l+m+mf}{0.3}\PY{p}{]}\PY{p}{)}\PY{p}{)}
          \PY{n}{ax}\PY{o}{.}\PY{n}{scatter}\PY{p}{(}\PY{n}{points}\PY{p}{[}\PY{l+m+mi}{0}\PY{p}{]}\PY{p}{,} \PY{n}{points}\PY{p}{[}\PY{l+m+mi}{1}\PY{p}{]}\PY{p}{,} \PY{n}{points}\PY{p}{[}\PY{l+m+mi}{2}\PY{p}{]}\PY{p}{,} \PY{n}{c}\PY{o}{=}\PY{n}{colors}\PY{p}{)} 
          \PY{n}{plt}\PY{o}{.}\PY{n}{show}\PY{p}{(}\PY{p}{)}
\end{Verbatim}


    \begin{center}
    \adjustimage{max size={0.9\linewidth}{0.9\paperheight}}{output_13_0.png}
    \end{center}
    { \hspace*{\fill} \\}
    
    \hypertarget{extrinsic-matrix}{%
\paragraph{Extrinsic Matrix}\label{extrinsic-matrix}}

The Extrinsic Matrix is used to represent the position of the camera
with respect to the the object coordinates. Earlier, the function
\texttt{camera\_coordinate\_system} was defined, and provided the
euclidian coordinates of the camera. This provides the cameras rotation,
with respect to the origin object. However, to properly specify the
camera position, a \emph{translation vector}, \texttt{t}, is needed to
where the camera is with respect to the object in the camera coordinate
system.

The extrinsic matrix, \texttt{E} is defined as:

\begin{equation*}
\mathbf{E} =
\ [R|t]\\
\end{equation*}

\texttt{R} is the rotation matrix, found via
\texttt{camera\_coordinate\_system}. Vector \texttt{t} is found via
\texttt{t\ =\ -RC}. Note that in the example provided, this dot product
must be done as \texttt{t\ =\ C\ @\ -R}, due to the fact that the
example code put the translation vector at the bottom of the matrix,
instead of the right side.

After finding \texttt{t}, we just append it to the end of the rotation
matrix to find \texttt{E}.

    \begin{Verbatim}[commandchars=\\\{\}]
{\color{incolor}In [{\color{incolor}109}]:} \PY{k}{def} \PY{n+nf}{ExtrinsicsMtx}\PY{p}{(}\PY{n}{cam}\PY{p}{)}\PY{p}{:}
              \PY{n}{xcam}\PY{p}{,} \PY{n}{ycam}\PY{p}{,} \PY{n}{zcam}\PY{p}{,} \PY{n}{origin}\PY{p}{,} \PY{o}{=} \PY{n}{camera\PYZus{}coordinate\PYZus{}system}\PY{p}{(}\PY{n}{cam}\PY{p}{)}
              
              \PY{c+c1}{\PYZsh{}rotation matrix}
              \PY{n}{R} \PY{o}{=} \PY{n}{np}\PY{o}{.}\PY{n}{asarray}\PY{p}{(}\PY{p}{[}\PY{n}{np}\PY{o}{.}\PY{n}{ravel}\PY{p}{(}\PY{n}{xcam}\PY{p}{)}\PY{p}{,} \PY{n}{np}\PY{o}{.}\PY{n}{ravel}\PY{p}{(}\PY{n}{ycam}\PY{p}{)}\PY{p}{,} \PY{n}{np}\PY{o}{.}\PY{n}{ravel}\PY{p}{(}\PY{n}{zcam}\PY{p}{)}\PY{p}{]}\PY{p}{)}\PY{o}{.}\PY{n}{T}
              
              \PY{c+c1}{\PYZsh{}translation vector}
              \PY{n}{t} \PY{o}{=} \PY{n}{origin} \PY{o}{@} \PY{o}{\PYZhy{}}\PY{n}{R}
              
              \PY{n}{M} \PY{o}{=} \PY{n}{np}\PY{o}{.}\PY{n}{vstack}\PY{p}{(}\PY{p}{[}\PY{n}{R}\PY{p}{,} \PY{n}{t}\PY{p}{]}\PY{p}{)}
              \PY{k}{return} \PY{n}{M}
\end{Verbatim}


    \hypertarget{intrinsic-matrix}{%
\paragraph{Intrinsic Matrix}\label{intrinsic-matrix}}

The Intrinsic Matrix models the image parameters after the image
``hits'' the lens. They way that the image hits the detector, or film,
is generally inverted and a different size. To accurately recreate the
acquired data, the intrinsic matrix must be solved. The matrix takes the
form:

\begin{equation*}
\mathbf{K} =  \begin{vmatrix}
\ F_x & 0 & 0 \\
\ s & F_y & 0 \\
\ cx & cy & 1 \\
\end{vmatrix}
\end{equation*}

\texttt{Fx} and \texttt{Fy} are the adjusted focal length, or the
distance from the lens to the detector. To convert the focal length, the
equation \texttt{Fx\ =\ fx(W/w)} is used, where \texttt{fx} is the focal
length and \texttt{W} is the camera width in pixels, and \texttt{w} is
the width of the detector.

\texttt{cx} and \texttt{cy} are the Principle Point Offset, which is
used to determine location of the principle point, or center of the
image input, with respect to the detector.

These values are very important for real world cameras, where the
parameters can vary signifcantly. Since this example is completely
virtual, there is no variability in this example.

    \begin{Verbatim}[commandchars=\\\{\}]
{\color{incolor}In [{\color{incolor}103}]:} \PY{k}{def} \PY{n+nf}{IntrinsicsMtx}\PY{p}{(}\PY{n}{cam}\PY{p}{)}\PY{p}{:}
              \PY{n}{cx} \PY{o}{=} \PY{l+m+mf}{0.5} \PY{o}{*} \PY{p}{(}\PY{n}{cam}\PY{p}{[}\PY{l+s+s1}{\PYZsq{}}\PY{l+s+s1}{width}\PY{l+s+s1}{\PYZsq{}}\PY{p}{]} \PY{o}{+} \PY{l+m+mi}{1}\PY{p}{)}
              \PY{n}{cy} \PY{o}{=} \PY{l+m+mf}{0.5} \PY{o}{*} \PY{p}{(}\PY{n}{cam}\PY{p}{[}\PY{l+s+s1}{\PYZsq{}}\PY{l+s+s1}{height}\PY{l+s+s1}{\PYZsq{}}\PY{p}{]} \PY{o}{+} \PY{l+m+mi}{1}\PY{p}{)}
              
              \PY{n}{fx} \PY{o}{=} \PY{n}{cam}\PY{p}{[}\PY{l+s+s1}{\PYZsq{}}\PY{l+s+s1}{focal\PYZus{}length}\PY{l+s+s1}{\PYZsq{}}\PY{p}{]} \PY{o}{*} \PY{n}{cam}\PY{p}{[}\PY{l+s+s1}{\PYZsq{}}\PY{l+s+s1}{width}\PY{l+s+s1}{\PYZsq{}}\PY{p}{]} \PY{o}{/} \PY{n}{cam}\PY{p}{[}\PY{l+s+s1}{\PYZsq{}}\PY{l+s+s1}{film\PYZus{}width}\PY{l+s+s1}{\PYZsq{}}\PY{p}{]}
              \PY{n}{fy} \PY{o}{=} \PY{n}{cam}\PY{p}{[}\PY{l+s+s1}{\PYZsq{}}\PY{l+s+s1}{focal\PYZus{}length}\PY{l+s+s1}{\PYZsq{}}\PY{p}{]} \PY{o}{*} \PY{n}{cam}\PY{p}{[}\PY{l+s+s1}{\PYZsq{}}\PY{l+s+s1}{height}\PY{l+s+s1}{\PYZsq{}}\PY{p}{]} \PY{o}{/} \PY{n}{cam}\PY{p}{[}\PY{l+s+s1}{\PYZsq{}}\PY{l+s+s1}{film\PYZus{}height}\PY{l+s+s1}{\PYZsq{}}\PY{p}{]}
              
              \PY{k}{return} \PY{n}{np}\PY{o}{.}\PY{n}{asarray}\PY{p}{(}\PY{p}{[}\PY{p}{[}\PY{n}{fx}\PY{p}{,} \PY{l+m+mi}{0}\PY{p}{,} \PY{l+m+mi}{0}\PY{p}{]}\PY{p}{,} \PY{p}{[}\PY{l+m+mi}{0}\PY{p}{,} \PY{n}{fy}\PY{p}{,} \PY{l+m+mi}{0}\PY{p}{]}\PY{p}{,} \PY{p}{[}\PY{n}{cx}\PY{p}{,} \PY{n}{cy}\PY{p}{,} \PY{l+m+mi}{1}\PY{p}{]}\PY{p}{]}\PY{p}{)}
\end{Verbatim}


    \hypertarget{camera-matrix}{%
\paragraph{Camera Matrix}\label{camera-matrix}}

Together, the extrinsic and intrinsic matrix create a full
representation of the camera geometry. The Camera matrix, \texttt{P} is
found with the equation:

\begin{equation*}
\mathbf{P} = 
\ K[R | -RC]\\
\end{equation*}

As we can see, it is simply the product of the intrinsic matrix
\texttt{K} and the extrinsic matrix \texttt{{[}R\ \textbar{}\ -RC{]}}.

    \begin{Verbatim}[commandchars=\\\{\}]
{\color{incolor}In [{\color{incolor}104}]:} \PY{k}{def} \PY{n+nf}{CameraMtx}\PY{p}{(}\PY{n}{cam}\PY{p}{)}\PY{p}{:}
              \PY{k}{return} \PY{n}{ExtrinsicsMtx}\PY{p}{(}\PY{n}{cam}\PY{p}{)} \PY{o}{@} \PY{n}{IntrinsicsMtx}\PY{p}{(}\PY{n}{cam}\PY{p}{)}
\end{Verbatim}


    \texttt{world2image} uses the camera matrix previously defined to
convert the 3D points defined earlier to the 2D coordinates interpreted
by the detector. The camera matrix defines the transform from 3D object
space to the 2D space of the detector. It makes converting the points as
easy as computing the dot product between each 3D point and the camera
matrix. However, in order for the 3D points to be able to be converted,
they must be homogenous. A \texttt{1} is appended to the end of each
point.

The result is a series of homogenous 2D points. To ensure that the
scaling is correct, they are divided by the third (homogenous) term.

    \begin{Verbatim}[commandchars=\\\{\}]
{\color{incolor}In [{\color{incolor}105}]:} \PY{k}{def} \PY{n+nf}{world2image}\PY{p}{(}\PY{n}{cam}\PY{p}{,} \PY{n}{points3d}\PY{p}{)}\PY{p}{:}
              \PY{n}{P} \PY{o}{=} \PY{n}{CameraMtx}\PY{p}{(}\PY{n}{cam}\PY{p}{)}\PY{p}{;}
              \PY{n}{A} \PY{o}{=} \PY{n}{points3d}\PY{o}{.}\PY{n}{T}
              
              \PY{c+c1}{\PYZsh{} make points homogenous}
              \PY{n}{B} \PY{o}{=} \PY{n}{np}\PY{o}{.}\PY{n}{ones}\PY{p}{(}\PY{p}{(}\PY{n+nb}{int}\PY{p}{(}\PY{n}{points}\PY{o}{.}\PY{n}{size}\PY{o}{/}\PY{l+m+mi}{3}\PY{p}{)}\PY{p}{,} \PY{l+m+mi}{1}\PY{p}{)}\PY{p}{)}
              \PY{n}{pt} \PY{o}{=} \PY{n}{np}\PY{o}{.}\PY{n}{hstack}\PY{p}{(}\PY{p}{[}\PY{n}{A}\PY{p}{,} \PY{n}{B}\PY{p}{]}\PY{p}{)}
              
              \PY{c+c1}{\PYZsh{}convert from 3D \PYZhy{}\PYZgt{} 2D}
              \PY{n}{pt} \PY{o}{=} \PY{n}{pt} \PY{o}{@} \PY{n}{P}
              
              \PY{c+c1}{\PYZsh{}scale the points }
              \PY{n}{x} \PY{o}{=} \PY{n}{pt}\PY{p}{[}\PY{p}{:}\PY{p}{,} \PY{l+m+mi}{0}\PY{p}{]} \PY{o}{/}\PY{n}{pt}\PY{p}{[}\PY{p}{:}\PY{p}{,}\PY{l+m+mi}{2}\PY{p}{]}
              \PY{n}{y} \PY{o}{=} \PY{n}{pt}\PY{p}{[}\PY{p}{:}\PY{p}{,} \PY{l+m+mi}{1}\PY{p}{]} \PY{o}{/} \PY{n}{pt}\PY{p}{[}\PY{p}{:}\PY{p}{,}\PY{l+m+mi}{2}\PY{p}{]}
              \PY{k}{return} \PY{n}{np}\PY{o}{.}\PY{n}{asarray}\PY{p}{(}\PY{p}{[}\PY{n}{x}\PY{p}{,} \PY{n}{y}\PY{p}{]}\PY{p}{)}
\end{Verbatim}


    This function draws the ``view'' from each camera. After finding the
location of each point in 2D, a circle is drawn at that point. This
represents the process for which a detector reconstructs the 3D world in
2D.

    \begin{Verbatim}[commandchars=\\\{\}]
{\color{incolor}In [{\color{incolor}106}]:} \PY{k}{def} \PY{n+nf}{setcolor}\PY{p}{(}\PY{n}{I}\PY{p}{,} \PY{n}{points2d}\PY{p}{,} \PY{n}{colors}\PY{p}{,} \PY{n}{R}\PY{p}{)}\PY{p}{:}
              \PY{n}{h}\PY{p}{,} \PY{n}{w}\PY{p}{,} \PY{n}{channels} \PY{o}{=} \PY{n}{I}\PY{o}{.}\PY{n}{shape}
              \PY{k}{for} \PY{n}{i} \PY{o+ow}{in} \PY{n+nb}{range}\PY{p}{(}\PY{n}{points2d}\PY{o}{.}\PY{n}{shape}\PY{p}{[}\PY{l+m+mi}{1}\PY{p}{]}\PY{p}{)}\PY{p}{:}
                  \PY{n}{x} \PY{o}{=} \PY{n}{points2d}\PY{p}{[}\PY{l+m+mi}{0}\PY{p}{,} \PY{n}{i}\PY{p}{]}
                  \PY{n}{y} \PY{o}{=} \PY{n}{points2d}\PY{p}{[}\PY{l+m+mi}{1}\PY{p}{,} \PY{n}{i}\PY{p}{]}
                  
                  \PY{n}{r1} \PY{o}{=} \PY{n+nb}{int}\PY{p}{(}\PY{n+nb}{max}\PY{p}{(}\PY{l+m+mi}{1}\PY{p}{,} \PY{n}{np}\PY{o}{.}\PY{n}{floor}\PY{p}{(}\PY{n}{y}\PY{o}{\PYZhy{}}\PY{n}{R}\PY{p}{)}\PY{p}{)}\PY{p}{)}
                  \PY{n}{r2} \PY{o}{=} \PY{n+nb}{int}\PY{p}{(}\PY{n+nb}{min}\PY{p}{(}\PY{n}{h}\PY{p}{,} \PY{n}{np}\PY{o}{.}\PY{n}{ceil}\PY{p}{(}\PY{n}{y}\PY{o}{+}\PY{n}{R}\PY{p}{)}\PY{p}{)}\PY{p}{)}
                  \PY{n}{c1} \PY{o}{=} \PY{n+nb}{int}\PY{p}{(}\PY{n+nb}{max}\PY{p}{(}\PY{l+m+mi}{1}\PY{p}{,} \PY{n}{np}\PY{o}{.}\PY{n}{floor}\PY{p}{(}\PY{n}{x}\PY{o}{\PYZhy{}}\PY{n}{R}\PY{p}{)}\PY{p}{)}\PY{p}{)}
                  \PY{n}{c2} \PY{o}{=} \PY{n+nb}{int}\PY{p}{(}\PY{n+nb}{min}\PY{p}{(}\PY{n}{w}\PY{p}{,} \PY{n}{np}\PY{o}{.}\PY{n}{ceil}\PY{p}{(}\PY{n}{x}\PY{o}{+}\PY{n}{R}\PY{p}{)}\PY{p}{)}\PY{p}{)}
                  
                  \PY{c+c1}{\PYZsh{}\PYZsh{} draw a circle at each point}
                  \PY{k}{for} \PY{n}{r} \PY{o+ow}{in} \PY{n+nb}{range}\PY{p}{(}\PY{n}{r1}\PY{p}{,} \PY{n}{r2}\PY{p}{)}\PY{p}{:}
                      \PY{k}{for} \PY{n}{c} \PY{o+ow}{in} \PY{n+nb}{range}\PY{p}{(}\PY{n}{c1}\PY{p}{,} \PY{n}{c2}\PY{p}{)}\PY{p}{:}
                          \PY{k}{if} \PY{p}{(}\PY{n}{r}\PY{o}{\PYZhy{}}\PY{n}{y}\PY{p}{)}\PY{o}{*}\PY{o}{*}\PY{l+m+mi}{2} \PY{o}{+} \PY{p}{(}\PY{n}{c}\PY{o}{\PYZhy{}}\PY{n}{x}\PY{p}{)}\PY{o}{*}\PY{o}{*}\PY{l+m+mi}{2} \PY{o}{\PYZlt{}} \PY{p}{(}\PY{n}{R}\PY{o}{*}\PY{o}{*}\PY{l+m+mi}{2}\PY{p}{)}\PY{p}{:}
                              \PY{n}{I}\PY{p}{[}\PY{n}{r}\PY{p}{,} \PY{n}{c}\PY{p}{,} \PY{p}{:}\PY{p}{]} \PY{o}{=} \PY{n}{colors}\PY{p}{[}\PY{n}{i}\PY{p}{,} \PY{p}{:}\PY{l+m+mi}{3}\PY{p}{]}
              \PY{k}{return} \PY{n}{I}
\end{Verbatim}


    The figures below show the 2D projections from camera 1 and camera 2.

    \begin{Verbatim}[commandchars=\\\{\}]
{\color{incolor}In [{\color{incolor}129}]:} \PY{n}{radius} \PY{o}{=} \PY{l+m+mi}{3}
          \PY{n}{px1} \PY{o}{=} \PY{n}{world2image}\PY{p}{(}\PY{n}{cam1}\PY{p}{,} \PY{n}{points}\PY{p}{)}
          \PY{n}{px2} \PY{o}{=} \PY{n}{world2image}\PY{p}{(}\PY{n}{cam2}\PY{p}{,} \PY{n}{points}\PY{p}{)}
          \PY{n}{I1} \PY{o}{=} \PY{n}{setcolor}\PY{p}{(}\PY{n}{np}\PY{o}{.}\PY{n}{ones}\PY{p}{(}\PY{p}{(}\PY{n}{cam1}\PY{p}{[}\PY{l+s+s1}{\PYZsq{}}\PY{l+s+s1}{height}\PY{l+s+s1}{\PYZsq{}}\PY{p}{]}\PY{p}{,} \PY{n}{cam1}\PY{p}{[}\PY{l+s+s1}{\PYZsq{}}\PY{l+s+s1}{width}\PY{l+s+s1}{\PYZsq{}}\PY{p}{]}\PY{p}{,} \PY{l+m+mi}{3}\PY{p}{)}\PY{p}{)}\PY{p}{,} \PY{n}{px1}\PY{p}{,} \PY{n}{colors}\PY{p}{,} \PY{n}{radius}\PY{p}{)}
          \PY{n}{I2} \PY{o}{=} \PY{n}{setcolor}\PY{p}{(}\PY{n}{np}\PY{o}{.}\PY{n}{ones}\PY{p}{(}\PY{p}{(}\PY{n}{cam2}\PY{p}{[}\PY{l+s+s1}{\PYZsq{}}\PY{l+s+s1}{height}\PY{l+s+s1}{\PYZsq{}}\PY{p}{]}\PY{p}{,} \PY{n}{cam1}\PY{p}{[}\PY{l+s+s1}{\PYZsq{}}\PY{l+s+s1}{width}\PY{l+s+s1}{\PYZsq{}}\PY{p}{]}\PY{p}{,} \PY{l+m+mi}{3}\PY{p}{)}\PY{p}{)}\PY{p}{,} \PY{n}{px2}\PY{p}{,} \PY{n}{colors}\PY{p}{,} \PY{n}{radius}\PY{p}{)}
          
          \PY{n}{f}\PY{p}{,} \PY{p}{(}\PY{n}{ax1}\PY{p}{,} \PY{n}{ax2}\PY{p}{)} \PY{o}{=} \PY{n}{plt}\PY{o}{.}\PY{n}{subplots}\PY{p}{(}\PY{l+m+mi}{1}\PY{p}{,} \PY{l+m+mi}{2}\PY{p}{,} \PY{n}{sharey}\PY{o}{=}\PY{k+kc}{True}\PY{p}{,} \PY{n}{figsize}\PY{o}{=}\PY{p}{(}\PY{l+m+mi}{8}\PY{p}{,} \PY{l+m+mi}{6}\PY{p}{)}\PY{p}{,} \PY{n}{dpi}\PY{o}{=}\PY{l+m+mi}{160}\PY{p}{)}
          \PY{n}{ax1}\PY{o}{.}\PY{n}{imshow}\PY{p}{(}\PY{n}{I1}\PY{p}{)}
          \PY{n}{ax2}\PY{o}{.}\PY{n}{imshow}\PY{p}{(}\PY{n}{I2}\PY{p}{)}
          \PY{n}{plt}\PY{o}{.}\PY{n}{show}\PY{p}{(}\PY{p}{)}
\end{Verbatim}


    \begin{center}
    \adjustimage{max size={0.9\linewidth}{0.9\paperheight}}{output_25_0.png}
    \end{center}
    { \hspace*{\fill} \\}
    
    \hypertarget{triangulation}{%
\subsubsection{Triangulation}\label{triangulation}}

Triangulation is the process of taking two seperate projections, their
camera matrices, and reconstruct the original scene.

\textbf{Triangulate} - The triangulate function takes the homogenous 2D
point locatons that would be acquired by the two cameras, and the camera
matrices. Each point is triangulated seperately, and the non-homogenous
estimated 3D point for each one is found.

\textbf{TriangulationOnePoint} - This method takes a the 2D location of
a point with respect to two cameras, and the camera matrix.
Triangulation can by defining a set of four linear equations, and using
SVD to determine a least squares solution to find the location of the
point in question.

The 2D points provided by each camera can be both found by a unknown
transformation, \texttt{X}. This relationship can be modeled by:

\begin{equation*}
\mathbf{x_1} = P_1X\
\end{equation*}

and \begin{equation*}
\mathbf{x_2} = P_2X\
\end{equation*} the lower case \texttt{x} represents the 2D homogenous
point, and the upper case \texttt{X} represents the 3D world point.

If an imaginary line was drawn between the center of each camera
detector and between each center and point X, a triangle is created.
This is where the term ``triangulation'' is coined from.

\texttt{P\_1} and \texttt{P\_2} can be combined to create, matrix A. By
setting \texttt{AX\ =\ 0}, the solution for A can be found via least
squares, which is the 3D point that most makes the equation closest to
0.

Then, the point is simply made non-homogenous, by dividing out the
fourth term.

    \begin{Verbatim}[commandchars=\\\{\}]
{\color{incolor}In [{\color{incolor}126}]:} \PY{k}{def} \PY{n+nf}{triangulate}\PY{p}{(}\PY{n}{points1}\PY{p}{,} \PY{n}{points2}\PY{p}{,} \PY{n}{P1}\PY{p}{,} \PY{n}{P2}\PY{p}{)}\PY{p}{:}
              \PY{n}{pixels} \PY{o}{=} \PY{n}{points1}\PY{o}{.}\PY{n}{shape}\PY{p}{[}\PY{l+m+mi}{1}\PY{p}{]}
              \PY{n}{points3d} \PY{o}{=} \PY{n}{np}\PY{o}{.}\PY{n}{zeros}\PY{p}{(}\PY{p}{(}\PY{n}{pixels}\PY{p}{,} \PY{l+m+mi}{3}\PY{p}{)}\PY{p}{)}
              
              \PY{k}{for} \PY{n}{i} \PY{o+ow}{in} \PY{n+nb}{range}\PY{p}{(}\PY{n}{pixels}\PY{p}{)}\PY{p}{:}
                  \PY{n}{points3d}\PY{p}{[}\PY{n}{i}\PY{p}{]} \PY{o}{=} \PY{n}{triangulationOnePoint}\PY{p}{(}\PY{n}{points1}\PY{p}{[}\PY{p}{:}\PY{p}{,}\PY{n}{i}\PY{p}{]}\PY{p}{,} \PY{n}{points2}\PY{p}{[}\PY{p}{:}\PY{p}{,}\PY{n}{i}\PY{p}{]}\PY{p}{,} \PY{n}{P1}\PY{o}{.}\PY{n}{T}\PY{p}{,} \PY{n}{P2}\PY{o}{.}\PY{n}{T}\PY{p}{)}
              \PY{k}{return} \PY{n}{points3d}
          
          \PY{k}{def} \PY{n+nf}{triangulationOnePoint}\PY{p}{(}\PY{n}{point1}\PY{p}{,} \PY{n}{point2}\PY{p}{,} \PY{n}{P1}\PY{p}{,} \PY{n}{P2}\PY{p}{)}\PY{p}{:}
              \PY{n}{A} \PY{o}{=} \PY{n}{np}\PY{o}{.}\PY{n}{zeros}\PY{p}{(}\PY{p}{(}\PY{l+m+mi}{4}\PY{p}{,} \PY{l+m+mi}{4}\PY{p}{)}\PY{p}{)}
              \PY{n}{A}\PY{p}{[}\PY{l+m+mi}{0}\PY{p}{:}\PY{l+m+mi}{2}\PY{p}{]} \PY{o}{=} \PY{p}{(}\PY{n}{np}\PY{o}{.}\PY{n}{outer}\PY{p}{(}\PY{n}{point1}\PY{p}{,} \PY{n}{P1}\PY{p}{[}\PY{l+m+mi}{2}\PY{p}{,}\PY{p}{:}\PY{p}{]}\PY{p}{)}\PY{p}{)} \PY{o}{\PYZhy{}} \PY{n}{P1}\PY{p}{[}\PY{l+m+mi}{0}\PY{p}{:}\PY{l+m+mi}{2}\PY{p}{,}\PY{p}{:}\PY{p}{]}
              \PY{n}{A}\PY{p}{[}\PY{l+m+mi}{2}\PY{p}{:}\PY{l+m+mi}{4}\PY{p}{]} \PY{o}{=} \PY{p}{(}\PY{n}{np}\PY{o}{.}\PY{n}{outer}\PY{p}{(}\PY{n}{point2}\PY{p}{,} \PY{n}{P2}\PY{p}{[}\PY{l+m+mi}{2}\PY{p}{,}\PY{p}{:}\PY{p}{]}\PY{p}{)}\PY{p}{)} \PY{o}{\PYZhy{}} \PY{n}{P2}\PY{p}{[}\PY{l+m+mi}{0}\PY{p}{:}\PY{l+m+mi}{2}\PY{p}{,}\PY{p}{:}\PY{p}{]}
              
              \PY{n}{u}\PY{p}{,} \PY{n}{s}\PY{p}{,} \PY{n}{V} \PY{o}{=} \PY{n}{np}\PY{o}{.}\PY{n}{linalg}\PY{o}{.}\PY{n}{svd}\PY{p}{(}\PY{n}{A}\PY{p}{)}
              \PY{n}{V} \PY{o}{=} \PY{n}{V}\PY{o}{.}\PY{n}{T}
              \PY{n}{X} \PY{o}{=} \PY{n}{V}\PY{p}{[}\PY{p}{:}\PY{p}{,}\PY{o}{\PYZhy{}}\PY{l+m+mi}{1}\PY{p}{]}
              \PY{n}{X} \PY{o}{/}\PY{o}{=} \PY{n}{X}\PY{p}{[}\PY{o}{\PYZhy{}}\PY{l+m+mi}{1}\PY{p}{]}
              
              \PY{k}{return} \PY{n}{X}\PY{p}{[}\PY{l+m+mi}{0}\PY{p}{:}\PY{l+m+mi}{3}\PY{p}{]}\PY{o}{.}\PY{n}{T}
\end{Verbatim}


    As we can see, this method works well in finding the original terms. To
model the reconstruction, the original points are drawn with a vertical
bar, and the reconstructed points are drawn with a horizontal bar.

    \begin{Verbatim}[commandchars=\\\{\}]
{\color{incolor}In [{\color{incolor}127}]:} \PY{n}{P1} \PY{o}{=} \PY{n}{CameraMtx}\PY{p}{(}\PY{n}{cam1}\PY{p}{)}
          \PY{n}{P2} \PY{o}{=} \PY{n}{CameraMtx}\PY{p}{(}\PY{n}{cam2}\PY{p}{)}
          \PY{n}{rec} \PY{o}{=} \PY{n}{triangulate}\PY{p}{(}\PY{n}{px1}\PY{p}{,} \PY{n}{px2}\PY{p}{,} \PY{n}{P1}\PY{p}{,} \PY{n}{P2}\PY{p}{)}
          \PY{n}{fig} \PY{o}{=} \PY{n}{plt}\PY{o}{.}\PY{n}{figure}\PY{p}{(}\PY{n}{figsize}\PY{o}{=}\PY{p}{(}\PY{l+m+mi}{8}\PY{p}{,} \PY{l+m+mi}{6}\PY{p}{)}\PY{p}{,} \PY{n}{dpi}\PY{o}{=}\PY{l+m+mi}{160}\PY{p}{)}
          \PY{n}{ax} \PY{o}{=} \PY{n}{fig}\PY{o}{.}\PY{n}{gca}\PY{p}{(}\PY{n}{projection}\PY{o}{=}\PY{l+s+s1}{\PYZsq{}}\PY{l+s+s1}{3d}\PY{l+s+s1}{\PYZsq{}}\PY{p}{)}
          \PY{n}{ax}\PY{o}{.}\PY{n}{scatter}\PY{p}{(}\PY{n}{points}\PY{p}{[}\PY{l+m+mi}{0}\PY{p}{]}\PY{p}{,} \PY{n}{points}\PY{p}{[}\PY{l+m+mi}{1}\PY{p}{]}\PY{p}{,} \PY{n}{points}\PY{p}{[}\PY{l+m+mi}{2}\PY{p}{]}\PY{p}{,} \PY{n}{c}\PY{o}{=}\PY{n}{colors}\PY{p}{,} \PY{n}{marker}\PY{o}{=}\PY{l+s+s2}{\PYZdq{}}\PY{l+s+s2}{|}\PY{l+s+s2}{\PYZdq{}}\PY{p}{)}
          \PY{n}{ax}\PY{o}{.}\PY{n}{scatter}\PY{p}{(}\PY{n}{rec}\PY{o}{.}\PY{n}{T}\PY{p}{[}\PY{l+m+mi}{0}\PY{p}{]}\PY{p}{,} \PY{n}{rec}\PY{o}{.}\PY{n}{T}\PY{p}{[}\PY{l+m+mi}{1}\PY{p}{]}\PY{p}{,} \PY{n}{rec}\PY{o}{.}\PY{n}{T}\PY{p}{[}\PY{l+m+mi}{2}\PY{p}{]}\PY{p}{,} \PY{n}{c}\PY{o}{=}\PY{n}{colors}\PY{p}{,} \PY{n}{marker}\PY{o}{=}\PY{l+s+s2}{\PYZdq{}}\PY{l+s+s2}{\PYZus{}}\PY{l+s+s2}{\PYZdq{}}\PY{p}{)}
          \PY{n}{plt}\PY{o}{.}\PY{n}{show}\PY{p}{(}\PY{p}{)}
\end{Verbatim}


    \begin{center}
    \adjustimage{max size={0.9\linewidth}{0.9\paperheight}}{output_29_0.png}
    \end{center}
    { \hspace*{\fill} \\}
    
    \hypertarget{conclusion}{%
\subsubsection{Conclusion}\label{conclusion}}

Camera geometry requires the combination of a couple of special
techniques and simple geometry. However, this model was overly
simplified, and the process of recreating it and learning the math
indicates to me that the application of these methods in real world
problems could express a variety of issues. Some of these issues include
variences in camera contrast, warping, and other optical/lens properties
that were unaccounted for in this example.


    % Add a bibliography block to the postdoc
    
    
    
    \end{document}
